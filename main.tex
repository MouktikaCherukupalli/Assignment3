\documentclass[journal,11pt,twocolumn]{IEEEtran}
\usepackage{gensymb}
\usepackage{amssymb}
\usepackage[cmex10]{amsmath}
\usepackage{amsthm}
\usepackage[export]{adjustbox}
\usepackage{bm}
\usepackage{longtable}
\usepackage{enumitem}
\usepackage{mathtools}
\usepackage{tikz}
\usepackage[breaklinks=true]{hyperref}
\usepackage{listings}
\usepackage{color}                                            %%
\usepackage{array}                                            %%
\usepackage{longtable}                                        %%
\usepackage{calc}                                             %%
\usepackage{multirow}                                         %%
\usepackage{hhline}                                           %%
\usepackage{ifthen}                                           %%
\usepackage{lscape}    
\usepackage{multicol}
% \usepackage{biblatex}
% \usepackage{enumerate}
\DeclareMathOperator*{\Res}{Res}
\renewcommand\thesection{\arabic{section}}
\renewcommand\thesubsection{\thesection.\arabic{subsection}}
\renewcommand\thesubsubsection{\thesubsection.\arabic{subsubsection}}
\renewcommand\thesectiondis{\arabic{section}}
\renewcommand\thesubsectiondis{\thesectiondis.\arabic{subsection}}
\renewcommand\thesubsubsectiondis{\thesubsectiondis.\arabic{subsubsection}}
\hyphenation{op-tical net-works semi-conduc-tor}
\def\inputGnumericTable{}                                 %%
\lstset{
frame=single, 
breaklines=true,
columns=fullflexible
}
\begin{document}
    \newtheorem{theorem}{Theorem}[section]
    \newtheorem{problem}{Problem}
    \newtheorem{proposition}{Proposition}[section]
    \newtheorem{lemma}{Lemma}[section]
    \newtheorem{corollary}[theorem]{Corollary}
    \newtheorem{example}{Example}[section]
    \newtheorem{definition}[problem]{Definition}
    \newcommand{\BEQA}{\begin{eqnarray}}
    \newcommand{\EEQA}{\end{eqnarray}}
    \newcommand{\define}{\stackrel{\triangle}{=}}
    \newcommand*\circled[1]{\tikz[baseline=(char.base)]{
        \node[shape=circle,draw,inner sep=2pt] (char) {#1};}}
    \bibliographystyle{IEEEtran}
    \providecommand{\mbf}{\mathbf}
    \providecommand{\pr}[1]{\ensuremath{\Pr\left(#1\right)}}
    \providecommand{\qfunc}[1]{\ensuremath{Q\left(#1\right)}}
    \providecommand{\sbrak}[1]{\ensuremath{{}\left[#1\right]}}
    \providecommand{\lsbrak}[1]{\ensuremath{{}\left[#1\right.}}
    \providecommand{\rsbrak}[1]{\ensuremath{{}\left.#1\right]}}
    \providecommand{\brak}[1]{\ensuremath{\left(#1\right)}}
    \providecommand{\lbrak}[1]{\ensuremath{\left(#1\right.}}
    \providecommand{\rbrak}[1]{\ensuremath{\left.#1\right)}}
    \providecommand{\cbrak}[1]{\ensuremath{\left\{#1\right\}}}
    \providecommand{\lcbrak}[1]{\ensuremath{\left\{#1\right.}}
    \providecommand{\rcbrak}[1]{\ensuremath{\left.#1\right\}}}
    \theoremstyle{remark}
    \newtheorem{rem}{Remark}
    
    
\title{Assignment 3}
\author{ Cherukupalli Sai Malini Mouktika\\\normalsize AI21BTECH11007 \\ \vspace*{10pt} \Large NCERT class10 Probability}



\maketitle

\textbf{15.2 example 5: }\\
Two players,Sangeeta and Reshma,play a tennis match. It is known that probability of Sangeeta winning the match is 0.62.What is the probability of Reshma winning the match?\\
\textbf{solution: }\\
Denote the probability of players winning the match by a random variable $X\in \cbrak{0, 1}$, where $X = 0$ denotes the probability of Sangeeta winning the match and $X = 1$ denotes the probability of Reshma winning the match.\\
Given,
\begin{align}
\pr{X = 0} =0.62
\end{align}
We know that these two events are mutually exclusive and exhaustive so,
\begin{align}
\pr{X = 0} + \pr{X = 1} = 1\\
\implies
\pr{X = 1} = 1- \pr{X = 0}\\
\implies
\pr{X = 1} = 1-0.62\\
\implies
\pr{X = 1} = 0.38
\end{align}
$\therefore$ the probability of Reshma winning the match is $0.38$

\end{document}